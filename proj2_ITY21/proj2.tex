%Druhy projekt do predmetu ITY 2020/2021.
%Autor: Alina Vinogradova
%email: xvinog00@stud.fit.vutbr.cz

\documentclass[11pt,a4paper,twocolumn]{article}

%packages
\usepackage[czech]{babel}
\usepackage[utf8]{inputenc}
\usepackage[IL2]{fontenc}
\usepackage[left=1.5cm,text={18cm, 25cm},top=2.5cm]{geometry}
\usepackage{times}
\usepackage{amsthm}
\usepackage{amssymb}
\usepackage{amsmath}
\usepackage{upgreek}
\usepackage{turnstile}
\usepackage{setspace}

\theoremstyle{definition}
\newtheorem{definition}{Definice}
\newtheorem{sentence}{Věta}

\begin{document}
	
\begin{titlepage}
		
	\begin{center}
		{\Huge\textsc{Fakulta informačních technologií \\ Vysoké učení technické v~Brně}}
		\vspace{\stretch{0.382}}
		{\LARGE \\ Typografie a~publikování -- 2. projekt \\ Sazba dokumentů a~matematických výrazů}
		\vspace{\stretch{0.618}}
	\end{center}

	{\Large\the\year\hfill Vinogradova Alina (xvinog00)}
		
\end{titlepage}
	
\section*{Úvod}
   
    V\,této\,úloze\,si\,vyzkoušíme\,sazbu\,titulní\,strany, matematických vzorců, prostředí a dalších textových struktur obvyklých pro technicky zaměřené texty (například rovnice (\ref{eq_1}) nebo Definice \ref{definition_rozsireny_zasobnikovy_automat} na straně \pageref{definition_rozsireny_zasobnikovy_automat}). Rovněž si vyzkoušíme používání odkazů \verb|\ref| a \verb|\pageref|.

    Na titulní straně je využito sázení nadpisu podle optického středu s využitím zlatého řezu. Tento postup byl probírán na přednášce. Dále je použito odřádkování se
    zadanou relativní velikostí 0.4 em a 0.3 em.

    V případě, že budete potřebovat vyjádřit matematickou
    konstrukci nebo symbol a nebude se Vám dařit jej nalézt
    v samotném \LaTeX u, doporučuji prostudovat možnosti balíku maker \AmS-\LaTeX.

\section{Matematický text}

    Nejprve \,se \,podíváme \,na \,sázení \,matematických \,symbolů
    a výrazů v plynulém textu včetně sazby definic a vět s využitím balíku \verb|amsthm|. Rovněž použijeme poznámku pod
    čarou s použitím příkazu \verb|\footnote|. Někdy je vhodné
    použít konstrukci \verb|\mbox{}|, která říká, že text nemá být
    zalomen.

\begin{definition}
\label{definition_rozsireny_zasobnikovy_automat}

    Rozšířený \,zásobníkový \,automat \emph{(RZA) je definován jako sedmice tvaru \(A = (Q, \Sigma, \Gamma, \updelta, q_0 , Z_0 , F)\), kde:}
    
\end{definition}

\renewcommand{\labelitemi}{$\bullet$}

\begin{itemize}

    \item \emph{\(Q\) je konečná množina} vnitřních (řídicích) stavů,

    \item \emph{\(\Sigma\) je konečná} vstupní abeceda,

    \item \emph{\(\Gamma\) je konečná} zásobníková abeceda,

    \item \emph{\(\updelta\) je} přechodová funkce \(Q \times (\Sigma \cup \{\epsilon\}) \times \Gamma ^* \to 2^{Q \times \Gamma ^*}\),

    \item \emph{\(q_0 \in Q\) je} počáteční stav, \emph{\(Z_0 \in \Gamma\) je} startovací symbol zásobníku \emph{a \(F \subseteq Q\) je množina} koncových stavů.

\end{itemize}

    Nechť \(P = (Q, \Sigma, \Gamma, \updelta, q_0 , Z_0 , F)\) je rozšířený zásobníkový automat. \emph{Konfigurací} nazveme trojici \emph{\((q, w, \alpha) \in Q \times \Sigma ^* \times \Gamma ^*\)}, kde \(q\) je aktuální stav vnitřního řízení, \(w\) je dosud nezpracovaná část vstupního řetězce a \(\alpha = Z_{i_1}Z_{i_2} \ldots Z_{i_k}\) je obsah zásobníku\footnote{\(Z_{i_1}\) je vrchol zásobníku}.

\subsection{Podsekce obsahující větu a odkaz}

\begin{definition}\label{retezec}

Řetězec \emph{\(w\)} nad abecedou \(\Sigma\) je přijat RZA \emph{A jestliže \((q_0, w, Z_0) \overset{*}{\underset{A}{\vdash}} (q_F,\epsilon, \gamma)\) pro nějaké \(\gamma \in \Gamma ^*\) a \(q_F \in F\). \,Množinu \(L(A) \,= \,\{w\:\:|\:\:w\) je přijat RZA A\} \(\subseteq \\\Sigma ^*\) nazýváme} jazyk přijímaný RZA \(A\).

\end{definition}

\bigskip
\bigskip

    Nyní si vyzkoušíme sazbu vět a důkazů opět s použitím
    balíku \verb|amsthm|.

\begin{sentence}
    \emph{Třída jazyků, které jsou přijímány ZA, odpovídá}
    bezkontextovým jazykům.
\end{sentence}

\begin{proof}
    V důkaze vyjdeme z Definice \ref{definition_rozsireny_zasobnikovy_automat} a \ref{retezec}.
\end{proof}
	
\section{Rovnice a odkazy}

    Složitější matematické formulace sázíme mimo plynulý
    text. Lze umístit několik výrazů na jeden řádek, ale pak je
    třeba tyto vhodně oddělit, například příkazem \verb|\quad|.
    
    $$
        \sqrt[i]{x^3_i} \;\;\;\, \text{kde} \; x_i \; \text{je} \; i\text{-té sudé číslo splňující} \;\;\; x^{x^{i^2}_i+2}_i \leq y^{x^4_i}_i
    $$
    
    V \,rovnici \,(\ref{eq_1}) \,jsou \,využity \,tři \,typy \,závorek s~různou explicitně definovanou velikostí.
    
    \begin{eqnarray}\label{eq_1}
        x & = & \bigg[ \Big\{ \big[ a + b\big] * c \Big\}^d \oplus 2 \bigg]^{3/2} \\
        y & = & \lim_{x\to\infty} \frac{\frac{1}{\log_{10} x}}{\sin^2x + \cos^2x} \nonumber
    \end{eqnarray}
    
    V této větě vidíme, jak vypadá implicitní vysázení limity $ \lim_{n\to\infty} f(n) $ v~normálním odstavci textu. Podobně je\,to\,i\,s\,dalšími symboly jako $ \prod^n_{i=1} 2^i$ či $\bigcap_{A \in \mathcal{B}} A$. V případě vzorců $ \lim\limits_{n\to\infty} f(n) $ a ${\prod\limits_{i=1}^{n} 2^i}$ jsme si vynutili méně úspornou sazbu příkazem \verb|\limits|.
    
    \begin{eqnarray}
      \int^a_b \, g(x) \, \mathrm{d}x & = & - \int\limits^b_a f(x) \, \mathrm{d}x
    \end{eqnarray}
    
\section{Matice}

    Pro sázení matic se velmi často používá prostředí \verb|array|
    a závorky (\verb|\left|, \verb|\right|).
    
    $$
        \left(
        \begin{array}{ccc}
             a - b & \widehat{\xi + \omega} & \pi \\
            \Vec{\mathbf{a}} & \overleftrightarrow{AC} & \hat{\beta}
        \end{array}
        \right) = 1 \iff \mathcal{Q} = \mathbb{R}
    $$
    
    $$
		\mathbf{A} =
		\left\|
		\begin{array}{cccc}
			a_{11} & a_{12} & \ldots & a_{1n} \\
			a_{21} & a_{22} & \ldots & a_{2n} \\
			\vdots & \vdots & \ddots & \vdots \\
			a_{m1} & a_{m2} & \ldots & a_{mn}
		\end{array}
		\right\| =
		    \left|
		    \begin{array}{rl}
		         t & u \\
		         v & w 
		    \end{array}
		    \right| = tw\!-\!uv
	$$
	
	Prostředí \verb|array| lze úspěšně využít i jinde.
	
	$$
		\binom{n}{k} =
		\left\{
		\begin{array}{cl}
			 0 & \text{pro } k < 0 \text{ nebo } k > n \\
			\frac{n!}{k! (n - k)!} & \text{pro } 0 \leq k \leq n.
		\end{array}
		\right.
	$$

\end{document}
